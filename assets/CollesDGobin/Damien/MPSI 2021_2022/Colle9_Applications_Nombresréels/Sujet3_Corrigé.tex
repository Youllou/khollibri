%%%%%%%%%%%%%%%%%%%%%%%%%%%%%%%%%%%   Packages de base   %%%%%%%%%%%%%%%%%%%%%%%%%%%%%%%%%%%


% Classe du document (book : chapitre inclus, pratique pour les gros cours (PASS, Tremplin), article (8pt) ou amsart (12pt) : pratique pour les cours sur une seule thématique)
\documentclass[a4paper, 11pt,openany]{article}% openany évite de commencer les chapitres sur une page forcément impaire (évite les pages blanches)

% Encodage, langue et font
\usepackage[utf8]{inputenc} % Pour les accents (conversion entre ces caractères accentués et les commandes d’accentuation)
\usepackage[french]{babel} % Français
\usepackage[T1]{fontenc} % Permet d'afficher et de prendre correctement en charge ces caractères accentués du point de vue du fichier de sortie

% Si on veut faire apparaître le texte avec des caractères plus larges
%\usepackage[lf]{Baskervaldx} % Texte en plus "gras"
%\usepackage[bigdelims,vvarbb]{newtxmath} % Lettre mathématiques en plus "gras"
%\usepackage[cal=boondoxo]{mathalfa} % Style pour les \mathscal
%\renewcommand*\oldstylenums[1]{\textosf{#1}}

% Mise en page
% Version automatique mais problème d'écart entre l'en-tête et le texte.
%\usepackage{fullpage} % Numérotation bas de page (et mise en page pleine).
\usepackage{setspace} % Interligne
\onehalfspacing % Interligne
\usepackage[a4paper]{geometry}% Package pour mise en page
\geometry{hscale=0.8,vscale=0.8,centering,headsep=0.5cm} % Marges, Headsep permet de ne pas coller le texte à l'en-tête

\setlength{\parindent}{0pt} % Supprimer les identations par défaut

\usepackage{fancyhdr} %Package permettant de mettre des en-têtes et des pieds de pages
\pagestyle{fancy}
\fancyhead{} % Enleve ce qui est présent par défaut
\fancyfoot{}% Enleve ce qui est présent par défaut
\usepackage{extramarks} % Pour écrire proprement le chapitre dans l'en-tête
%\renewcommand{\chaptermark}[1]{\markboth{\chaptername\ \thechapter.\ #1}{}} % Redéfinition du chapitre pour écriture dans l'en-tête sous la forme "Chapitre n. Nom du chapitre"
%\renewcommand{\chaptermark}[1]{\markboth{\thechapter.\ #1}{}} % Forme "n. Nom du chapitre"
%\renewcommand{\chaptermark}[1]{\markboth{#1}{}} % Forme "Nom du chapitre"
\renewcommand{\headrulewidth}{0.6pt} % Epaisseur trait en haut
\renewcommand{\footrulewidth}{0.6pt} % Epaisseur trait en bas
\setlength{\headheight}{15pt}
% Placer les informations où on veut (à noter : E: Even page ; O: Odd page ; L: Left field ; C: Center field ; R: Right field ; H: Header ; F: Footer) :
\fancyhead[LO,LE]{\slshape  \textbf{MPSI2}}
\fancyhead[RO,RE]{\slshape \textbf{Corrigé - Colle 9 (Sujet 3)}}
%\fancyhead[CO,CE]{---Draft---}
\fancyfoot[C]{\thepage}
%\fancyfoot[LO, LE] {\slshape Damien GOBIN}
%\fancyfoot[RO, RE] {\slshape Année 2021-2022}

% Couleurs
\usepackage{xcolor}
\definecolor{BleuTresFonce}{rgb}{0.0,0.0,0.250}
%\definecolor{bleu}{rgb}{0.36, 0.54, 0.66}
\definecolor{bleu}{rgb}{0.47, 0.62, 0.8}
%\definecolor{vert}{rgb}{0.33, 0.42, 0.18}
%\definecolor{vert}{rgb}{0.52, 0.73, 0.4}
\definecolor{vert}{rgb}{0.56, 0.74, 0.56}
% Liens hypertextes
\usepackage[colorlinks,final,hyperindex]{hyperref}
\hypersetup{
	pdftex,
	linkcolor=BleuTresFonce,
	citecolor=BleuTresFonce,
	filecolor=BleuTresFonce,
	urlcolor=BleuTresFonce,
	pdftitle=TemplateCours,
	pdfauthor=Damien Gobin,
	pdfsubject=,
	pdfkeywords=
}


% Inclure des figures
\usepackage{graphicx} % Permet d'utiliser includegraphics
\usepackage{pdfpages} % Permet d'insérer des fichiers pdf

% Utiliser des commentaires
\usepackage{comment}
%\excludecomment{sol} %commenter cette ligne si on veut faire apparaitre les solutions d'exercices

%%%%%%%%%%%%%%%%%%%%%%%%%%%%%%%%%%%   Packages mathématiques  %%%%%%%%%%%%%%%%%%%%%%%%%%%%%%%%%%%

% Packages mathématiques de base
%\usepackage{amsfonts} % Permet de taper des ensembles 
\usepackage{amsmath} % Permet de taper des maths (contient cleveref)
\usepackage{amsthm} % Permet de définir une environnement pour les théorèmes
\usepackage{amssymb} % Donne accès a plus de symboles mathématiques (charge de façon automatique amsfonts)
% \usepackage{mathrsfs} % Ecriture ronde type mathcal
%\usepackage{bbold} % Permet d'avoir l'indicatrice mais change les textbb
\usepackage{stmaryrd} % Pour les intervalles entiers [[ ]]
\usepackage{pifont} % Pour avoir accès à plus de caractères

%Utiliser les symboles jeu de cartes
\DeclareSymbolFont{extraup}{U}{zavm}{m}{n}
\DeclareMathSymbol{\varheart}{\mathalpha}{extraup}{86}
\DeclareMathSymbol{\vardiamond}{\mathalpha}{extraup}{87}



% Référence dans le document
\usepackage[noabbrev,capitalize]{cleveref}

% Tableau
\usepackage{tabularx} % Tableau
\usepackage{multirow} % Pour créer des tableaux en subdivisant les lignes
\usepackage{diagbox} %Pour créer des crois dans les tableaux à double entrées
\usepackage{enumitem} %Permet de scinder une énumération et de reprendre au numéro suivant
\usepackage{multicol} % Création de document en colonne avec plusieurs colonnes
\multicolsep=5pt % supprime l'espace vertical

% Modification des itemizes
\setitemize{label=$\bullet$} % Utilise le package enumitem et met des points au lieu des tirets

% Ecrire des algorithmes en "français" (exemple issu du cours d'optimisation 2020/2021)
\usepackage[linesnumbered, french, frenchkw,ruled]{algorithm2e}
% Exemple :
%\begin{center}
%\begin{algorithm}
%\Entree{Un graphe $G = (V,E)$, $|V| = n$, $|E| = m$ et pour chaque arête $e$ de $E$ son poids $c(e)$;}
%\Sortie{Un arbre (ou une forêt) maximal $A = (V,F)$ et de poids minimum ;} 
%Trier et renuméroter les arêtes de $G$ dans l'ordre croissant de leur poids : $c(e_1) \leqslant c(e_2) \leqslant ... \leqslant c(e_m)$ \;
%Poser $F := \emptyset$ et $k = 0$\;
%\Tq{$k<m$ et $|F| < n-1$}{Si $e_{k+1}$ ne forme pas de cycle avec $F$ alors $F := F \cup \{ e_k \}$\;
%$k := k +1 $\;}
%\caption{Algorithme de Kruskal théorique (1956)}
%\end{algorithm}
%\end{center}

% Inclure du code Python avec coloration syntaxique et bloc gris (exemple issu du TP2 d'optimisation 2020/2021)
%\usepackage{xcolor} % Déjà importé plus haut
\usepackage{listings}
\lstset{backgroundcolor=\color{darkWhite},literate={á}{{\'a}}1 {é}{{\'e}}1 {í}{{\'i}}1 {ó}{{\'o}}1 {ú}{{\'u}}1{Á}{{\'A}}1 {É}{{\'E}}1 {Í}{{\'I}}1 {Ó}{{\'O}}1 {Ú}{{\'U}}1{à}{{\`a}}1 {è}{{\`e}}1 {ì}{{\`i}}1 {ò}{{\`o}}1 {ù}{{\`u}}1{À}{{\`A}}1 {È}{{\'E}}1 {Ì}{{\`I}}1 {Ò}{{\`O}}1 {Ù}{{\`U}}1{ä}{{\"a}}1 {ë}{{\"e}}1 {ï}{{\"i}}1 {ö}{{\"o}}1 {ü}{{\"u}}1{Ä}{{\"A}}1 {Ë}{{\"E}}1 {Ï}{{\"I}}1 {Ö}{{\"O}}1 {Ü}{{\"U}}1{â}{{\^a}}1 {ê}{{\^e}}1 {î}{{\^i}}1 {ô}{{\^o}}1 {û}{{\^u}}1{Â}{{\^A}}1 {Ê}{{\^E}}1 {Î}{{\^I}}1 {Ô}{{\^O}}1 {Û}{{\^U}}1{œ}{{\oe}}1 {Œ}{{\OE}}1 {æ}{{\ae}}1 {Æ}{{\AE}}1 {ß}{{\ss}}1{ű}{{\H{u}}}1 {Ű}{{\H{U}}}1 {ő}{{\H{o}}}1 {Ő}{{\H{O}}}1{ç}{{\c c}}1 {Ç}{{\c C}}1 {ø}{{\o}}1 {å}{{\r a}}1 {Å}{{\r A}}1{€}{{\EUR}}1 {£}{{\pounds}}1}
\lstdefinestyle{stylepython}{        language=Python,        basicstyle=\ttfamily,    commentstyle=\color{green},    keywordstyle=\color{blue},    stringstyle=\color{olive},    numberstyle=\tiny,        numbers=left,        stepnumber=1,         numbersep=5pt}
% Exemple :
%\begin{lstlisting}[style=stylepython]
%import networkx as nx #Cette bibliothèque permettra de manipuler des graphes
%import matplotlib.pyplot as plt #Cette bibliothèque permettra de les représenter
%import numpy as np
%\end{lstlisting}

%Package permettant de tracer des graphes, des courbes, etc.. (exemple issu du cours d'optimisation 2020/2021)
\usepackage{tikz}
\usepackage{tikz-cd}
\usepackage{tkz-tab} % Pour les tableaux de signes
%\usetikzlibrary{shapes,backgrounds}
\usetikzlibrary{
	decorations.pathmorphing,
	arrows,
	arrows.meta,
	calc,
	shapes,
	shapes.geometric,
	decorations.pathreplacing,}
\tikzcdset{arrow style=tikz, diagrams={>=stealth}}
\tikzset{>=stealth'}
\tikzstyle{rond}=[draw,circle,thick,fill=white]
% Exemple courbe :
%\begin{center}
%\begin{tikzpicture}
%	\def\shift{.5}
%	\def\xmax{6}
%	\def\ymax{7}
%	\draw[->] (-\shift,0) -- (\xmax+\shift,0) node[below] {$x$}; 
%	\foreach \x in {1,...,\xmax}{ \pgfmathtruncatemacro\xbis{100*\x}
%		\draw (\x,0) -- (\x,-.3*\shift) node[below] {$\scriptstyle{\xbis}$};
%	}
%	% axe ordonnees
%	\draw[->] (0,-\shift) -- (0,\ymax+\shift) node[left] {$y$};  
%	\foreach \y in {1,...,\ymax}{ \pgfmathtruncatemacro\ybis{100*\y}
%		\draw (0,\y) -- (-.3*\shift,\y) node[left] {$\scriptstyle{\ybis}$};
%	}
%	\draw[fill=yellow] (0,0) -- (3,0) -- (0,6) -- cycle;
%	
%	\draw[thick, purple] plot [domain=-.15:6.15, variable=\t] (\t,6-\t);
%	\node[above, text=purple] (A) at (6.9,.1) {$x+y = 600$};
%	%%
%	\draw[thick, purple] plot [domain=-.1:3.1, variable=\t] (\t,6-2*\t);
%	\node[right, text=purple] (B) at (1.1,5) {$2x+y = 600$};
%	%%
%	\draw[thick, blue] plot [domain=-.1:3.85, variable=\t] (\t,6-1.6*\t);
%	%%
%	\draw[fill=blue] (0,6) circle (3pt);
%	\node[right, text=blue] (B) at (0.1,6.3) {Solution optimale};
%\end{tikzpicture}
%\end{center}
% Exemple graphe :
%\begin{center}
%\begin{tikzpicture}[scale=0.9]
%    \draw[thick] (-1.5,0) node[rond] {$1$} coordinate (A) --
%        ++(3,0) node[rond] {$2$} coordinate (B) --
%        ++(-1,-1) node[rond] {$4$}  coordinate (D) --
%        ++(-1,0) node[rond] {$3$}  coordinate (C) --
%        ++(0,-1) node[rond] {$5$}  coordinate (E) --
%        ++(1,0) node[rond] {$6$}  coordinate (F) --
%        ++(1,-1) node[rond] {$8$}  coordinate (H) --
%        ++(-3,0) node[rond] {$7$}  coordinate (G) -- (A)
%        (A) -- (C)
%        (G) -- (E)
%        (D) -- (F)
%        (B) -- (H);
%\end{tikzpicture}
%\end{center}

%%%%%%%%%%%%%%%%%%%%%%%%%%%%%%%%%%%   Environnements  %%%%%%%%%%%%%%%%%%%%%%%%%%%%%%%%%%%

%Différents types à donner aux environnements théorèmes, définitions, etc...
%[theorem] permet de numéroter par rapport à la section en cours
%\renewcommand{\theexem}{\empty{}} permet de ne pas numéroterNe numérote pas les exemples

%\usepackage{thmtools} % Permet de créer un environnement de théorème avec des cadres
%\declaretheorem[thmbox=L]{boxtheorem L}
%\declaretheorem[thmbox=M]{boxtheorem M}
%\declaretheorem[thmbox=S]{boxtheorem S}
%
%%\usepackage[dvipsnames]{xcolor}
%%\declaretheorem[shaded={bgcolor=Lavender,textwidth=12em}]{BoxI}
%\declaretheorem[shaded={rulecolor=black,rulewidth=1pt}]{BoxII}
%
%
%\usepackage[leftmargin=0.1865cm,rightmargin=0.6cm]{thmbox}
% Voici ici pour les options https://ctan.mines-albi.fr/macros/latex/contrib/thmbox/thmbox.pdf

%% Ecrit de cette façon tout le monde a le même style : Titre en droit et texte en italique très léger.
%\newtheorem[L,leftmargin=0.1865cm,rightmargin=0.1865cm,cut=true]{thm}{Théorème}[subsection]
%%\renewcommand{\thetheorem}{\empty{}} 
%\newtheorem[M,leftmargin=0.1865cm,rightmargin=0.1865cm,cut=true]{propo}[thm]{Proposition}
%%\renewcommand{\thepropo}{\empty{}} 
%\newtheorem[M,leftmargin=0.1865cm,rightmargin=0.1865cm,cut=true]{prop}[thm]{Propriété}
%%\renewcommand{\theprop}{\empty{}} 
%\newtheorem[M,leftmargin=0.1865cm,rightmargin=0.1865cm,cut=true]{coro}[thm]{Corollaire}
%\newtheorem[M,leftmargin=0.1865cm,rightmargin=0.1865cm,cut=true]{lem}[thm]{Lemme}
%\newtheorem[M,leftmargin=0.1865cm,rightmargin=0.1865cm,cut=true]{defi}[theorem]{Définition}
%%\renewcommand{\thedefinition}{\empty{}}
%\newtheorem[S,leftmargin=0.1865cm,rightmargin=0.1865cm,cut=true]{exem}{Exemple}
%\renewcommand{\theexem}{\empty{}} 
%\newtheorem[S,leftmargin=0.1865cm,rightmargin=0.1865cm,cut=true]{exo}{Exercice}
%\renewcommand{\theexo}{\empty{}}
%\newtheorem[S,leftmargin=0.1865cm,rightmargin=0.1865cm,cut=true]{sol}{Solution}
%\renewcommand{\thesol}{\empty{}} 
%\newtheorem[S,leftmargin=0.1865cm,rightmargin=0.1865cm,cut=true]{hypo}{Hypothesis}[section]
%\newtheorem[S,leftmargin=0.1865cm,rightmargin=0.1865cm,cut=true]{remark}[thm]{Remarque}%[section]
%\newtheorem{dem}{Démonstration}
%\renewcommand{\thedem}{\empty{}} 
%\newtheorem[S]{nota}{Notation}[section]

\theoremstyle{plain}
\newtheorem{theorem}{Théorème}[subsection]
%\renewcommand{\thetheorem}{\empty{}} 
\newtheorem{propo}[theorem]{Proposition}
%\renewcommand{\thepropo}{\empty{}} 
\newtheorem{prop}[theorem]{Propriété}
%\renewcommand{\theprop}{\empty{}} 
\newtheorem{coro}[theorem]{Corollaire}
\newtheorem{lemma}[theorem]{Lemme}

\theoremstyle{definition}
\newtheorem{definition}[theorem]{Définition}
%\renewcommand{\thedefinition}{\empty{}}
\newtheorem{exem}{Exemple}
\renewcommand{\theexem}{\empty{}} 
\newtheorem{cours}{Question de cours}
\renewcommand{\thecours}{\empty{}} 
\newtheorem{exo}{Exercice}
%\renewcommand{\theexo}{\empty{}} 
\newtheorem{sol}{Solution de l'exercice}
%\renewcommand{\thesol}{\empty{}} 
\newtheorem{hypo}{Hypothesis}[section]
\newtheorem{remark}[theorem]{Remarque}%[section]


\theoremstyle{remark}
\newtheorem{dem}{Démonstration}
\renewcommand{\thedem}{\empty{}} 
\newtheorem{nota}{Notation}[section]

%%%%%%%%%%%%%%%%%%%%%%%%%%%%%%%%%%%   Raccourcis   %%%%%%%%%%%%%%%%%%%%%%%%%%%%%%%%%%%
 
% Ensembles
\newcommand{\R}{\mathbb{R}} 
\newcommand{\Q}{\mathbb{Q}}
\newcommand{\C}{\mathbb{C}}
\newcommand{\Z}{\mathbb{Z}}
\newcommand{\K}{\mathbb{K}}
\newcommand{\N}{\mathbb{N}}
\newcommand{\D}{\mathbb{D}}


% Lettres calligraphiques
\newcommand{\calP}{\mathcal{P}} 
\newcommand{\calF}{\mathcal{F}} 
\newcommand{\calD}{\mathcal{D}} 
\newcommand{\calQ}{\mathcal{Q}} 
\newcommand{\calT}{\mathcal{T}} 

% Noyau et image
\newcommand{\im}{\text{Im}} 
\newcommand{\noy}{\text{Ker}} 
\newcommand{\card}{\text{Card}} 

%Epsilon
\newcommand{\vare}{\varepsilon} 


% Fonctions usuelles
%\newcommand{\un}{\mathbb{1}} % Indicatrice en utilisant le package \usepackage{bbold} mais celui modifie les mathbb donc on définit l'indicatrice à la main
\def\un{{\mathchoice {\rm 1\mskip-4mu l} {\rm 1\mskip-4mu l}
{\rm 1\mskip-4.5mu l} {\rm 1\mskip-5mu l}}} % Indicatrice


\title{Corrigé - Colle 9 (Sujet 3)}
\author{MPSI2\\
Année 2021-2022}
%\author{Mathématiques P.A.S.S. 1}
\date{30 novembre 2021}


\begin{document}


   \maketitle
%      \rule{\linewidth}{0.5mm}
      \rule{\linewidth}{0.5mm}
%  \begin{center}
%  %    \rule{\linewidth}{0.5mm}\\[0.4cm]
%       { \huge \bfseries Mathématiques pour le P.A.S.S 1\\[0.4cm] }
%    \rule{\linewidth}{0.5mm}\\[4cm]
%     \end{center}

\begin{cours}
Montrer que $\R$ est archimédien.
\end{cours}


\begin{exo}
%Difficulté : 1/5
%Chapitre : Applications
Soient $E$ et $F$ deux ensembles et soit $f:E \to F$. Soient également $A$ et $B$ deux parties de $E$.
\begin{enumerate}
\item Démontrer que$ A \subset B \Rightarrow f(A) \cup f(B)$. La réciproque est-elle vraie?
\item Démontrer que $f(A \cap B) \subset f(A) \cap f(B)$. L'inclusion réciproque est-elle vraie?
\item Démontrer que $f(A \cup B)=f(A) \cup f(B)$.
\end{enumerate}
\end{exo}

\begin{sol}
\begin{enumerate}
\item  Prenons $y \in f(A)$. Alors il existe $x \in A$ tel que $y=f(x)$. Mais alors, $x \in B$ et donc $y \in f(B)$. Ceci prouve l'inclusion $f(A) \subset f(B)$. La réciproque n'est pas toujours vraie. Prenons $E=\{1,2\}$, $F=\{1\}$, $f:E \to F$ définie par $f(1)=1$, $f(2)=1$, $A=\{1\}$ et $B=\{2\}$. Alors $f(A)=f(B)=\{1\}$ alors que pourtant $A$ n'est pas inclus dans $B$.
\item On a $A \cap B \subset A$, et donc $f(A\cap B) \subset f(A)$. De même, $f(A \cap B) \subset f(B)$, et donc $f(A \cap B) \subset f(A) \cap f(B)$. L'inclusion réciproque est fausse, ce que l'on constate en prenant exactement le même exemple.
\item On a $A \subset A \cup B$ et donc $f(A) \subset f(A \cup B)$. De même, $f(B)\subset f(A \cup B)$ et donc $f(A)\cup f(B)\subset f(A\cup B)$.\\
Réciproquement, si $y \in f(A \cup B)$, alors il existe $x \in A \cup B$ tel que $y=f(x)$. Mais si $x \in A$, on a $y \in f(A) \subset f(A) \cup f(B)$ et de même, si $x \in B$, on a $y \in f(B) \subset f(A) \cup f(B)$. Dans tous les cas, on a prouvé que $y \in f(A)\cup f(B)$ et donc l'inclusion $f(A \cup B) \subset f(A) \cup f(B)$. 
\end{enumerate}
\end{sol}


\begin{exo}
%Difficulté : 2/5
%Chapitre : Nombres réels
Soit $A$ une partie non-vide et bornée de $\R$. On note $B=\{|x-y|, \ (x,y)\in A^2\}$.
\begin{enumerate}
\item Justifier que $B$ est majorée.
\item On note $\delta(A)$la borne supérieure de cet ensemble. Prouver que $\delta(A)=\sup(A)-\inf(A)$. \end{enumerate}
\end{exo}

\begin{sol}
\begin{enumerate}
\item Soient $(x,y) \in A^2$ et soit $M \in \R$ tel que $x \in A$ implique $|x| \leqslant M$. Alors on a 
\[ |x-y|\leqslant |x|+|y| \leqslant 2M,\]
ce qui prouve que $B$ est majoré. 
\item Posons $m=\inf(A)$ et $M=\sup(A)$. Soient $(x,y)\in R^2$. Alors on a 
\[ m \leqslant x \leqslant M \quad \text{et} \quad -M \leqslant -y \leqslant -m \quad \Rightarrow \quad -(M-m)\leqslant x-y \leqslant M-m\]
d'où on tire $|x-y|\leqslant M-m$. On en déduit donc que $M-m$ est un majorant de $B$ et que $\delta (A)\leqslant M-m$. Pour prouver l'autre inégalité, on fixe $\varepsilon >0$, et on construit un élément $b \in B$ tel que $b>M-m-\varepsilon$. Pour cela, on sait qu'il existe $(x,y) \in A^2$ tels que 
\[ x \leqslant M- \frac{\varepsilon}{2} \quad \text{et}\quad  y \leqslant m + \frac{\varepsilon}{2}.\] Alors $x-y\geqslant M-m- \varepsilon$, ce qui est le résultat voulu. On a donc bien $\delta(A)=\sup(A)-\inf(A)$.
\end{enumerate}
\end{sol}

\begin{exo}
%Difficulté : 2/5
%Chapitre : Applications
On considère la fonction
$\displaystyle{h : x \mapsto \frac{2x+1}{x+2}}$.
\begin{enumerate}
\item Déterminer l'ensemble de définition $\calD_h$ de $h$.
\item Déterminer $\im(h) = h(\calD_h)$ et montrer que $h$ est bijective de $\calD_h$ dans $\im(h)$.
\item Donner son application réciproque.
\end{enumerate}
\end{exo}

\begin{sol}
\begin{enumerate}
\item $\calD_h = \R \setminus \{-2\}$ car c'est le seul endroit où le dénominateur s'annule (et le numérateur ne s'y annule pas).
\item Si $y \in \im(h)$, alors il existe $x \in \R \setminus \{-2\}$ tel que $h(x) = y$. Or,
\[h(x) = y \quad \Leftrightarrow \quad \frac{2x+1}{x+2} = y \quad \Leftrightarrow \quad 2x+1 = y(x+2) \quad \Leftrightarrow \quad (2-y)x = 2y-1.\]
Ainsi, si $y \neq 2$, $x = \frac{2y-1}{2-y}$ est un antécédent de $y$. Ceci montre que $\R \setminus \{2\} \subset \im(h)$. Afin de montrer que c'est une égalité, montrons à présent que $2$ n'a pas d'antécédent par $f$. Supposons par l'absurde qu'il existe $x$ tel que $h(x) = 2$. Alors,
\[ h(x) = 2 \quad \Leftrightarrow \quad \frac{2x+1}{x+2} =2 \quad \Leftrightarrow \quad 2x+1 = 2(x+2) \quad \Leftrightarrow \quad 1 = 4\]
ce qui est absurde. On a ainsi montré que $\im(h) = \R \setminus \{2\}$.\\
Montrons à présent que $h$ est bijective de $\calD_h$ dans $\im(h)$. Cela revient à montré que $h$ est injective. Soit $(x,y) \in (\R \setminus \{-2\})^2$ tel que $h(x) = h(y)$. Alors,
\[h(x) = h(y) \Leftrightarrow \quad \frac{2x+1}{x+2} = \frac{2y+1}{y+2} \quad \Leftrightarrow \quad (2x+1)(y+2) = (2y+1)(x+2).\]
Ainsi,
\[ h(x) = h(y) \quad \Leftrightarrow \quad 2xy + 4x + y + 2 = 2xy + 4y + x + 2 \quad \Leftrightarrow \quad   3(x-y)  = 0 .\]
On en déduit que $h(x) = h(y)$ si et seulement si $x =y$. $h$ est donc bijective de $\calD_h$ dans $\im(h)$.
\item Nous avons déjà montré que pour tout $y \neq 2$,
\[h(x) = y \quad \Leftrightarrow \quad x = \frac{2y-1}{2-y}.\]
L'application réciproque de $h$ est donc donnée par
\[  \begin{array}{ccccc}
h^{-1} & : & \R \setminus \{2\} & \to & \R \setminus \{-2\} \\
 & & x & \mapsto & \frac{2x-1}{2-x}
\end{array}.\]
\end{enumerate}
\end{sol}

\begin{exo}
%Difficulté : 2/5
%Chapitre : Applications
Soit $f :E \to F$ et $A \subset F$. Montrer que $ A \subset f^{-1}(f(A))$. Trouver un contre-exemple pour l'autre inclusion. Que peut-on dire si $f$ est de plus injective ?
\end{exo}

\begin{sol}
\begin{enumerate}
 \item Soit $x\in A$. Alors, $x\in f^{-1}(f(A))$ équivaut à $f(x)\in f(A)$. Or $f(x)\in f(A)$ puisque $x\in A$ donc on a bien $A \subset f^{-1}(f(A)$. 
  \item Donnons un contre-exemple lorsque $f$ n'est pas injective. On peut choisir $f\colon \mathbb R \to \mathbb R$ définie par $f(x)=\cos(x)$. Pour $A=[0,2\pi]$ on a $f(A)=[-1,1]$ et $f^{-1}f(A))=\mathbb R$. Donc $f^{-1}(f(A))$ n'est pas inclu dans $A$.
 \item Supposons que $f$ est injective. Soit $x\in f^{-1}(f(A))$. Alors, $f(x)\in f(A)$. Donc il existe $x' \in A$ tel que $f(x)=f(x')$. Par injectivité de $f$, on a donc $x=x'$. D'où $x\in A$ et $f^{-1}(f(A)) \subset A$. Ainsi, lorsque $f$ est injective on a  $f^{-1}(f(A)) = A$.
\end{enumerate}
\end{sol}
   

\begin{exo}
%Difficulté : 4/5
%Chapitre : Applications
Soient $E$ un ensemble, $\mathcal{P}(E)$ l'ensemble de ses parties, et $A$ et $B$ deux parties de $E$. On définit $f: \mathcal{P}(E) \to \mathcal{P}(A) \times  \mathcal{P}(B)$ par $f(X) = (X\cap A,X\cap B)$.
\begin{enumerate}
\item Montrer que $f$ est injective si et seulement si $A \cup B=E$.
\item Montrer que $f$ est surjective si et seulement si $A \cap B = \emptyset$.
\item Donner une condition nécessaire et suffisante sur $A$ et $B$ pour que $f$ soit bijective. Donner dans ce cas la bijection réciproque. 
\end{enumerate}
\end{exo}

\begin{sol}
\begin{enumerate}
\item Pour démontrer le sens direct, on raisonne par contraposée: si $A \cup B \neq E$, on prend $x \in E \setminus (A \cup B)$ et $X=\{x\}$. Alors $f(X)=(X \cap A,X \cap B)=( \emptyset , \emptyset)$ car $x$ n'appartient ni à $A$ ni à $B$. D'autre part, $f(\emptyset)=(\emptyset,\emptyset)$. Donc $f(X)=f(\emptyset)$ alors que $X \neq \emptyset$ : $f$ n'est pas injective.\\
Pour le sens réciproque, remarquons que pour tout $X \subset E$, puisque $A \cup B=E$, on a \[X=X \cap E=X \cap (A \cup B)=(X\cap A) \cup (X\cap B).\] Ainsi, si $X,X' \subset E$ sont tels que $f(X)=f(X')$, c'est-à-dire $X \cap A=X' \cap A$ et $X \cap B=X' \cap B$, on a 
\[ X=(X\cap A) \cup (X \cap B)=(X' \cap A) \cup (X' \cap B)=X'.\] Ainsi, $f$
est injective.
\item Supposons d'abord que $f$ est surjective et prenons $x \in A$. Alors il existe $X \subset E$ tel que $f(X)=(\{x\},\emptyset)$. Alors, on a $X \cap B= \emptyset$ et $x \in X \cap A$. Ainsi, $x \in X$ et donc $x \notin B$. Ainsi, on a $A \cap B= \emptyset$.\\
Réciproquement, si $A \cap B= \emptyset$, et prenons $A' \subset A$ et $B' \subset B$. Alors, posons $X=A' \cup B'$. Puisque $A \cap B= \emptyset$, on a $X \cap A=A'$ et $X \cap B=B'$ et donc $f(X)=(A',B')$ : $f$ est surjective.
\item D'après les questions précédentes, on a $f$ bijective si et seulement si $A \cup B=E$ et $A \cap B= \emptyset$, i.e. si $(A,B)$ est une partition de $E$. La bijection réciproque a été établie à la question précédente et est donnée par $(A',B') \mapsto A' \cup B'$.
\end{enumerate}
\end{sol}















\end{document}
