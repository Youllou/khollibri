Commençons par remarquer que $f$ est $\pi$-périodique, car $sin(\pi+x)^2=sin(x)^2$ et $cos(\pi+x)^2=cos(x)^2$. Il suffit donc d'étudier $f$ sur $[0,\pi]$. De plus, on a également $f(\pi-x)=f(x)$ et donc il suffit d'étudier $f$ sur $\left[0, \frac{\pi}{2} \right]$. Soit $u(x)=\int_0^x \arcsin(\sqrt{t}) \,dt$ et $v(x)=\int_0^x \arccos(\sqrt{t}) \,dt$. Puisque les fonctions $\arcsin$ et $\arccos$ sont continues sur $[0,1]$, $u$ et $v$ sont de classe $C^1$ sur $[0,1]$, avec $u'(x)=\arcsin(\sqrt{x})$ et $v'(x)=\arccos(\sqrt{x})$. De plus, $f(x)=u(\sin(x)^2)+v(cos(x)^2)$. Par composition, $f$ est de classe $C^1$ sur $\left[0,\frac{\pi}{2} \right]$, par conséquent sur $\R$, et sa dérivée est 
\[ f'(x)=2\sin(x) \cos(x) \arcsin(\sqrt{\sin(x)^2}) - 2\sin(x) \cos(x) \arccos(\sqrt{\cos(x)^2}).\]
On peut se restreindre à $x$ dans l'intervalle $\left[ 0 ,\frac{\pi}{2} \right]$], et pour $x$ dans cet intervalle, tout se passe bien, à savoir $\sqrt{\sin(x)^2}=\sin(x)$, $\sqrt{\cos(x)^2}=\cos(x)$ et $\arcsin(\sin(x))=x$, $\arccos(\cos(x))=x$. Il vient :
\[ f'(x)=2x\sin(x)\cos(x)-2x\sin(x)\cos(x)=0.\] Ainsi, la fonction $f$ est constante sur $\R$. De plus, on a, pour tout $x \in \R$, 
\[ f(x)=f \left( \frac{\pi}{4} \right) =  \int_0^{\frac{1}{2}} \arccos(\sqrt{t}) + \arcsin(\sqrt{t}) \, dt= \int_0^{\frac{1}{2}} \frac{\pi}{2} \, dt = \frac{\pi}{4},\]
puisque l'on sait que pour tout $x \in [-1,1]$, $\arcsin(x)+\arccos(x)= \frac{\pi}{2}$.