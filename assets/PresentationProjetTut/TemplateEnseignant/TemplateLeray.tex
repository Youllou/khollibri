\documentclass[12pt,a4paper]{amsart}

\newcommand{\classe}{MP2I }
\newcommand{\annee}{2021/2022}
\newcommand{\NbColle}{11}
\newcommand{\jourColle}{14/12/21} 

\usepackage{titlesec}
% \usepackage{titletoc} % for the table of contents
\titleformat{name=\section}{}{{
	{\large \sc Lycée Clémenceau -- \classe -- Colle \NbColle}
}}{0.4em}{\medskip\centering\scshape}

\titleformat{name=\subsection}{}{{
	\jourColle\ -- Feuille \thesection \ -- J. \textsc{Leray} -- \textsc{Étudiant$\cdot$e} : \hspace{4cm}
}}{0.4em}{\bigskip\centering\scshape}

\usepackage{geometry}
\geometry{ 
	hmargin=2cm, % left and right margin
	vmargin=2cm % top and bottom margin
}
\usepackage{setspace}
\onehalfspacing

% Package 
\usepackage[utf8]{inputenc}
\usepackage[french]{babel}

% Font 
\usepackage[T1]{fontenc}
\usepackage[lf]{Baskervaldx} % lining figures
\usepackage[bigdelims,vvarbb]{newtxmath} % math italic letters from Nimbus Roman
\usepackage[cal=boondoxo]{mathalfa} % mathcal from STIX, unslanted a bit
\renewcommand*\oldstylenums[1]{\textosf{#1}}

\usepackage[normalem]{ ulem }
\usepackage{soul}

\usepackage{graphicx}

%%% MATHS
\usepackage{amsmath,amsfonts,amscd,amsthm,mathrsfs} %bug avec amssymb ???
\usepackage{stmaryrd}
\usepackage{latexsym}

%%% SYMBOLES
\usepackage{savesym}
\savesymbol{checkmark}
\usepackage{dingbat}
\usepackage{manfnt}


\usepackage{enumitem}
\setlist[enumerate,1]{
    label=\arabic*.,
    leftmargin =2em
}
\setlist[enumerate,2]{
    label=\alph*),
    leftmargin =2em
}


% COLORS
\usepackage{xcolor}
\definecolor{Chocolat}{rgb}{0.36, 0.2, 0.09}
\definecolor{BleuTresFonce}{rgb}{0.215, 0.215, 0.36}
\definecolor{bleu}{rgb}{0.47, 0.62, 0.8}

\usepackage[colorlinks,final,hyperindex]{hyperref}
\hypersetup{
	pdftex,
	linkcolor=Chocolat,
	citecolor=BleuTresFonce,
	filecolor=BleuTresFonce,
	urlcolor=BleuTresFonce,
	pdftitle=\classe-\annee-\NbColle,
	pdfauthor=Johan Leray,
	pdfsubject=,
	pdfkeywords=
}

\usepackage[noabbrev,capitalize]{cleveref}
\setlength{\parindent}{0pt}

\usepackage{tikz}
\usepackage{tikz-cd}
\usepackage{tkz-tab} % Pour les tableaux de signes
%\usetikzlibrary{shapes,backgrounds}
\usetikzlibrary{
	decorations.pathmorphing,
	arrows,
	arrows.meta,
	calc,
	shapes,
	shapes.geometric,
	decorations.pathreplacing,}
\tikzcdset{arrow style=tikz, diagrams={>=stealth}}
\tikzset{>=stealth'}

%%%%%%%%%% ENVIRONNEMENT %%%%%%%%%%%
\theoremstyle{definition}
\newtheorem*{question}{\sc Question de cours}
\newtheorem{exercice}{\sc Exercice}[section]
% \newtheorem*{solution}{\sc Solution} 

\usepackage{thmtools}

\declaretheoremstyle[
	headfont=\scshape\scriptsize,
	notefont=\rmfamily\scriptsize,
	bodyfont=\rmfamily\scriptsize,
]{stylesolution}
\declaretheorem[style=stylesolution,numbered=no,name=Solution]{solution}

%%%%%%%%%% MACROS %%%%%%%%%


\usepackage{xparse}       	% Parseur
\ExplSyntaxOn
% #1 = command to use
% #2 = optional prefix
% #3 = list
\NewDocumentCommand{\createbunch}{ m O{} m }
 {
  \clist_map_inline:nn { #3 } { \cs_new_protected:cpn { #2 ##1 } { #1 { ##1 } } }
 }
\ExplSyntaxOff
% LETTERS
\createbunch{\mathbb}   [bb]{A,B,C,D,E,F,G,H,I,J,K,L,M,N,O,P,Q,R,S,T,U,V,W,X,Y,Z}
\createbunch{\mathbf}   [bf]{A,B,C,D,E,F,G,H,I,J,K,L,M,N,O,P,Q,R,S,T,U,V,W,X,Y,Z}
\createbunch{\mathcal}  [cal]{A,B,C,D,E,F,G,H,I,J,K,L,M,N,O,P,Q,R,S,T,U,V,W,X,Y,Z}
\createbunch{\mathscr}  [scr]{A,B,C,D,E,F,G,H,I,J,K,L,M,N,O,P,Q,R,S,T,U,V,W,X,Y,Z}
\createbunch{\mathsf}   [sf]{A,B,C,D,E,F,G,H,I,J,K,L,M,N,O,P,Q,R,S,T,U,V,W,X,Y,Z}
\createbunch{\mathrm}   [rm]{A,B,C,D,E,F,G,H,I,J,K,L,M,N,O,P,Q,R,S,T,U,V,W,X,Y,Z,a,b,c,d,e,f,g,h,i,j,k,l,m,o,p,q,r,s,t,u,v,w,x,y,z}
\createbunch{\mathfrak} [frak]{A,B,C,D,E,F,G,H,I,J,K,L,M,N,O,P,Q,R,S,T,U,V,W,X,Y,Z}

\renewcommand{\Im}{\mathrm{Im}}
\renewcommand{\deg}{\mathrm{deg}}
\newcommand{\eps}{\varepsilon}
\newcommand{\llbr}{\llbracket}
\newcommand{\Rbr}{\Rbracket}
\newcommand{\Aff}{\mathbb{A}^2_{\bbR}}

\newcommand{\N}{\mathbb{N}}
\newcommand{\Z}{\mathbb{Z}}
\newcommand{\R}{\mathbb{R}}
\newcommand{\C}{\mathbb{C}}
\newcommand{\too}{\longrightarrow}
\newcommand{\mapstoo}{\longmapsto}

\DeclareMathOperator{\Arcsin}{Arcsin}
\DeclareMathOperator{\Arccos}{Arccos}
\DeclareMathOperator{\Arctan}{Arctan}
\DeclareMathOperator{\ch}{ch}
\DeclareMathOperator{\sh}{sh}
\renewcommand{\tanh}{\mathrm{th}} 

%%%% PROBAS
\newcommand{\pr}{\bbP}
\newcommand{\Esp}{\bbE}
\newcommand{\Var}{\bbV}
\newcommand{\Cov}{\bbC\mathrm{ov}}

\DeclareMathOperator{\Card}{Card}
\DeclareMathOperator{\id}{id}
\DeclareMathOperator{\Ker}{Ker}


\newcommand{\Johan}[1]{\textcolor{purple}{ #1}}
\newcommand{\johan}[1]{\textcolor{purple}{ #1}}


%%%%%%%%%%%%%%   Title & Authors    %%%%%%%%%%%%

\title{Lycée Clémenceau -- MP2I}
\date{\today}

\author{Johan Leray}
% \email{johan.leray@univ-nantes.fr}
% \address{Laboratoire de Mathématiques Jean Leray -- Université de Nantes}
\thanks{}


%%%%%%%%%% AVEC OU SANS CORRECTION %%%%%%%%%%
\usepackage{comment}
\excludecomment{solution}


\pagestyle{empty}

\begin{document}

\thispagestyle{empty}

%%%%%%%%%%%%%%%%% FEUILLE 1 %%%%%%%%%%%%%%%%
\section{}
\subsection{}

\begin{question}
	Montrer que la loi définie par 
	\[
		a \star b = \frac{a+b}{1+ab} 
	\]
	est une loi de composition interne sur $]-1,1[$, associative, commutative. Préciser son neutre et exprimer le symétrique de $x$ en fonction de $x$ pour tout $x\in ]-1,1[$.
\end{question}


\begin{exercice}
	On considère les applications ensembles suivantes 
	\[
		\begin{array}[]{rccc}
			f \colon & \N & \too & \N \\
			& n & \mapstoo & 2n
		\end{array}
		\quad 
		\mbox{ et }
		\quad 
		\begin{array}[]{rccl}
			g \colon & \N & \too & \N \\
			& n & \mapstoo & 
			\left\{ 
				\begin{array}{cc}
					\frac{n}{2} & \mbox{ si $n$ est pair,} \\
					0 & \mbox{ sinon.}
				\end{array}
			\right.
		\end{array}
	\]
	\begin{enumerate}
		\item Déterminer $g\circ f$ et $f\circ g$.
		\item L'application $f$ est-elle injective ? Surjective ? Bijective ?
		\item L'application $g$ est-elle injective ? Surjective ? Bijective ?
	\end{enumerate}
\end{exercice}

\begin{exercice}
	On considère la fonction  
	\[
		\begin{array}{rccc}
			f \colon & \R & \too & \R \\
			& x & \mapstoo & \frac{x}{1 + |x|}
		\end{array}
	\]
	\begin{enumerate}
		\item Montrer que $f$ réalise une bijection de $\R$ vers un sous-ensemble de $\R$ à déterminer.
		\item Déterminer la bijection réciproque.
		\item Démontrer que pour tous réels $x$ et $y$, on a l'inégalité suivante
		\[
			\frac{|x+y|}{1 +|x+y|} \leqslant \frac{|x|}{1+|x|} + \frac{|y|}{1+|y|} \ .
		\]
	\end{enumerate}
\end{exercice}

\begin{exercice}\leavevmode
	\begin{enumerate}
		\item Montrer que pour tout réel $x$, on a $\lfloor x+1 \rfloor = \lfloor x \lfloor +1$.
		\item Montrer que pour tous réels $x$ et $y$, on a $\lfloor x \rfloor + \lfloor y \rfloor \leqslant \lfloor x +y\lfloor $.
		\item Montrer que pour tous réels $x$ et $y$, on a $\lfloor x \rfloor + \lfloor y \rfloor + \lfloor x+y \rfloor \leqslant \lfloor 2x \rfloor + \lfloor 2y \rfloor$.
	\end{enumerate}
\end{exercice}


%%%%%%%%%%%%%%%%% FEUILLE 2 %%%%%%%%%%%%%%%%
~
\newpage
\section{}
\subsection{}

\begin{question}
	Unicité de l'élément neutre sous réserve d'existence d'une loi interne. Unicité sous réserve d'existence du symétrique d'un élément pour une loi interne associative.
\end{question}

\begin{exercice}
	%Difficulté : 1/5
	%Chapitre : Applications
	Soit $f\colon x \mapsto \dfrac{x}{x+1}$. Exprimer $f^n(x) = \underbrace{f\circ f \circ \ldots \circ f}_{n\text{ fois}}(x)$ pour tout entier naturel $n \geqslant 1$.
	\end{exercice}
	
	\begin{solution}
	Montrons par récurrence que la propriété $f^n(x) = \frac{x}{nx+1}$ est vraie pour tout $n \geqslant 1$.
	\begin{itemize}
	\item \textbf{Initialisation.} Pour $n=1$ le résultat provient immédiatement de la définition de $f$.
	\item \textbf{Hérédité.} Soit $n \geqslant 1$ fixé. Supposons que le résultat est vrai au rang $n$, i.e. que $f^n(x) = \frac{x}{nx+1}$. Alors,
	\[ f^{n+1}(x) = f(f^n(x)) = f \left( \frac{x}{nx+1} \right) = \frac{\frac{x}{nx+1}}{\frac{x}{nx+1} +1 } = \frac{x}{x+(nx+1)} = \frac{x}{(n+1)x+1}.\]
	Le résultat est donc vrai au rang $n+1$.
	\item \textbf{Conclusion.} On a donc démontré par récurrence que pour tout $n \geqslant 1$, $f^n(x) = \frac{x}{nx+1}$.
	\end{itemize}
	\end{solution}
	
	
\begin{exercice}
%Difficulté : 1/5
%Chapitre : Ensembles
On définit les ensembles
\[ K = [2,5] \times [-1,4] \quad \text{et} \quad D = \{(x,y) \in \R^2, y \leqslant x \}.\]
\begin{enumerate}
\item Représenter dans le plan le domaine des points $M(x,y)$ appartenant à $K$ et le domaine des points $N(x,y)$ appartenant à $D$.
\item L'implication suivante est-elle vraie ?
\[ \forall (x,y) \in \R^2, \quad (x,y) \in K \quad \Rightarrow \quad (x + 1,y - 1)\in D.\]
\item La réciproque est-elle vraie ?
\item Montrer que $K \cap D \neq \emptyset$.
\end{enumerate}
\end{exercice}
	
	\begin{solution}
	\begin{enumerate}
	\item On a les représentations suivantes (où on a mis en gras les "bords" inclus dans le domaine) :
	  \begin{center}
	\begin{tikzpicture}[scale=0.8]
		  \draw [very thick, ->] (-1,0)--(6,0);
		 \draw [very thick, ->] (0,-2)--(0,5);
		 \draw [help lines] (-1,-2) grid (6,5);
	\fill[bleu!100,opacity=0.2] (2,-1) rectangle (5,4);
		 \draw [very thick,bleu!150] (2,-1)--(5,-1);
		 \draw [very thick,bleu!150] (2,4)--(5,4);
		 \draw [very thick,bleu!150] (2,-1)--(2,4);
		 \draw [very thick,bleu!150] (5,-1)--(5,4);
		 \draw (1.2,-0.3) node {$1$};
		 \draw (-0.2,1.2) node {$1$};
		 \draw (2.5,-2.5) node {\textbf{Ensemble $K$}};
		\end{tikzpicture}
	\hspace{1.5cm} \text{et} \hspace{1.5cm}
	\begin{tikzpicture}[scale=0.8]
		  \draw [very thick, ->] (-1,0)--(6,0);
		 \draw [very thick, ->] (0,-2)--(0,5);
		 \draw [help lines] (-1,-2) grid (6,5);
	\fill[bleu!100,opacity=0.2]  (-1,-1) -- (-1,-2) -- (6,-2)-- (6,5) -- (5,5) -- cycle;
		 \draw [very thick,bleu!150] (-1,-1)--(5,5);
		 \draw (1.2,-0.3) node {$1$};
		 \draw (-0.2,1.2) node {$1$};
		 \draw (2.5,-2.5) node {\textbf{Ensemble $D$}};
		\end{tikzpicture}
		\end{center}
	\item Cette implication est vraie. En effet, si $(x,y) \in K$, alors $2 \leqslant x \leqslant 5$ et $-1 \leqslant y \leqslant 4$. Ainsi, $3 \leqslant x+1 \leqslant 6$ et $ -2 \leqslant y-1\leqslant 3$ et donc $y-1 \leqslant x+1$ ce qui signifie que $(x+1,y-1) \in D$.
	\item La réciproque est fausse. En effet, par exemple $(10,9) \in D$ mais $(10 - 1,9 + 1) = (9,10) \notin K$.
	\item $K \cap D \neq \emptyset$ car $(3,3) \in K \cap D$.
	\end{enumerate}
	\end{solution}

\begin{exercice}
	Soit $x$ un nombre réel. Déterminer
	\[
		\lim_{n\to +\infty} \frac{\lfloor x\rfloor+\lfloor 2x\rfloor+ \ldots = \lfloor nx\rfloor}{n^2} \ .
	\]
\end{exercice}
\begin{solution}

\end{solution}

%%%%%%%%%%%%%%%%% FEUILLE 3 %%%%%%%%%%%%%%%%
~
\newpage
\section{}
\subsection{}

\begin{question}
	Soit $D$ une partie de $\R$. Alors $D$ est convexe si et seulement si $D$ est un intervalle.
\end{question}

\begin{exercice}
	%Difficulté : 2/5
	%Chapitre : Applications
	On considère
	\[ \begin{array}{ccccc}
	f & : & \R & \to & \C \\
	 & & x & \mapsto & \frac{1+ix}{1-ix}
	\end{array} .\]
	\begin{enumerate}
	 \item Montrer que $f$ est bien définie. $f$ est-elle injective ? $f$ est-elle surjective ?
	 \item Déterminer $f^{-1}(\mathbb R)$.
	 \item Montrer que $f(\mathbb R)=\{z\in\mathbb C,\ |z|=1,\ z\neq-1\}$.
	\end{enumerate}
\end{exercice}

\begin{exercice}
	Soient $E$ un ensemble, $\mathcal{P}(E)$ l'ensemble des parties de $E$ et $A,B \in \mathcal{P}(E)$ deux parties de $E$. On définit l'application  
	\[
		\begin{array}[]{rccc}
			f \colon & \mathcal{P}(E) & \too & \mathcal{P}(A) \times \mathcal{P}(B) \\
			& X & \mapstoo & (X\cap A, X\cap B)
		\end{array}
	\]
	\begin{enumerate}
		\item Montrer que $f$ est injective si et seulement si $A\cup B = E$.
		\item Montrer que $f$ est surjective si et seulement si $A \cap B = \emptyset$.
		\item Donner une condition nécessaire et suffisante sur $A$ et $B$ pour que $f$ soit bijective. Dans ce cas, donner sa bijection réciproque.
	\end{enumerate}
\end{exercice}

\begin{exercice}
	Montrer que pour tout entier $n\in\N$, et pour tout réel $x$, on a 
	\[
		\left\lfloor \frac{\lfloor nx \rfloor}{n}	\right\rfloor = \lfloor x \rfloor \ .
	\]
\end{exercice}


\end{document}