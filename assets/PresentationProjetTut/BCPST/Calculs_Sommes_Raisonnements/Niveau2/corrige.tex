On commence par remarquer que la première racine n'est définie que pour $x(x-2) \geqslant 0$, i.e. pour $x \in ]- \infty,0]\cup [2,+\infty[$ et la deuxième racine carrée n'est quant à elle définie seulement pour $2x - 3 \geqslant 0$, i.e. pour $x \geqslant \frac{3}{2}$. Ainsi, notre inéquation n'a de sens que pour $x \in [2,+\infty[$. On utilise à présent la multiplication par la quantité conjuguée (qui est strictement positive (car elle est positive ou nulle et nous venons de voir que les deux racines ne peuvent s'annuler simultanément) et ne change donc pas l'inégalité) :
\[\sqrt{x^2 - 2x} - \sqrt{2x-3} < 0 \quad \Leftrightarrow \quad \frac{(\sqrt{x^2 - 2x} - \sqrt{2x-3})(\sqrt{x^2 - 2x} + \sqrt{2x-3})}{\sqrt{x^2 - 2x} + \sqrt{2x-3}} < 0.\]
Or,
\[ (\sqrt{x^2 - 2x} - \sqrt{2x-3})(\sqrt{x^2 - 2x} + \sqrt{2x-3}) = x^2 - 2x + -2x + 3 = x^2 - 4x + 3.\]
Ainsi,
\[\sqrt{x^2 - 2x} - \sqrt{2x-3} < 0 \quad \Leftrightarrow \quad \frac{x^2 - 4x + 3}{\sqrt{x^2 - 2x} + \sqrt{2x-3}} < 0.\]
Or, le dénominateur est strictement positif, donc
\[\sqrt{x^2 - 2x} - \sqrt{2x-3} < 0 \quad \Leftrightarrow \quad x^2 - 4x + 3 < 0.\]
Or, le trinôme $x^2 - 4x + 3$ a pour discriminant $\Delta = 16 - 12 =4$ et possède donc deux racines distinctes
\[x_1 = \frac{4 - 2}{2} = 1 \quad \text{et} \quad x_2 = \frac{4 + 2}{2} = 3.\]
Ainsi, il est strictement négatif si et seulement si $x \in ]1,3[$. On rappelle que l'on travaille pour $x \geqslant 2$ et conclut donc que
\[\sqrt{x^2 - 2x} - \sqrt{2x-3} < 0 \quad \Leftrightarrow \quad x \in [2,3[.\]