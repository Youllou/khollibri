On commence par rappeler que $\sin(x)^2 = 1 -\cos(x)^2$. Ainsi,
\[ \sin(x)^2 + \frac{3\sqrt{2}}{2} \cos(x) =2 \ \Leftrightarrow \  1 -\cos(x)^2 + \frac{3\sqrt{2}}{2} \cos(x) =2 \ \Leftrightarrow \  \cos(x)^2 - \frac{3\sqrt{2}}{2} \cos(x) +1 = 0. \]
On a donc un trinôme en $X = \cos(x)$. De plus,
\[ \Delta = \frac{9 \times 2}{4} - 4 =  \frac{9}{2} - 4 = \frac{1}{2} > 0\]
donc il y a deux solutions,
\[ X_1 = \frac{\frac{3\sqrt{2}}{2} - \frac{1}{\sqrt{2}}}{2} = \frac{\sqrt{2}}{2} \quad \text{et} \quad X_2 = \frac{\frac{3\sqrt{2}}{2} + \frac{1}{\sqrt{2}}}{2} = \sqrt{2}.\]
Or, $\cos(x) = \sqrt{2}$ n'a pas de solution. Finalement, on a donc
\[ \sin(x)^2 + \frac{3\sqrt{2}}{2} \cos(x) =2 \quad \Leftrightarrow \quad \cos(x) = \frac{\sqrt{2}}{2} \quad \Leftrightarrow \quad x = \pm \frac{\pi}{4} + 2 k \pi.\]